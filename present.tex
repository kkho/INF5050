\documentclass[12pt]{beamer}
\usepackage[latin1]{inputenc}
\usepackage{fancybox}
\usepackage{soul}
\usepackage{wasysym}
\usecolortheme{wolverine}

\title[]{INF5050 - Protocols and routing in internet}
\subtitle[]{Multiprotocol Label Switching (MPLS) /\newline
			Generalized Multiprotocol Label Switching (GMPLS) }

% \includegraphics[width=1.5cm]{Puzzle.eps}} % note not shown


\author{Khiem-Kim Ho Xuan -- kkho@ifi.uio.no, \newline
		Mattias H�heim Johnsen -- mattiahj@ifi.uio.no}
\date{1. March 2013}
\begin{document}

\begin{frame}
  \titlepage
\end{frame}

\begin{frame}
  \frametitle{Outline}
  \begin{itemize}
  \item Background
  \item MPLS Fundamentals and Terminology
  \item Control and Forwarding Plane
  \item Generalized MPLS
  \item GMPLS Recovery techniques
  \item Summary
  \item Resources
  \end{itemize}
\end{frame}

\begin{frame}
  \frametitle{Background}

  \begin{itemize}
	\item What is MPLS?
		\begin{itemize}
			\item Mechanism that directs data from one network node to the next based on on path labels rather than network addresses.
			\item with such mechanism, we avoid lookups in a routing table
			\item MPLS switches packets (IP packets) instead of routing packets to transport the data
		\end{itemize}
	
	\item Why MPLS?
		\begin{itemize}
			\item Provide a highly scalable mechanism that was topology driven rather than flow driven
			\item Load balance traffic to utilize network bandwidth efficiently
			\item Allow core routers/networking devices to switch packets based on a simplified header
			\item Remove the complexity and overhead of network managements (Assemble and reassemble IP packets)
		\end{itemize}			
	\end{itemize} 

  \bigskip

\end{frame}

\begin{frame}
  \frametitle{MPLS was conceived, why?}
  	\begin{itemize}
  		\item The shortest path routing protocols like IS-IS and OSPF
  			\begin{itemize}
  				\item Did not take capacity characteristics into account while
  				making the routing decisions
  				\item The outcome is, segmentation over the network which leads
  				to congestion, while others remain under-utilized.
  			\end{itemize}
  		\item MPLS reduces the complexity and redundancies by adding new network
  		functionalities.
  	\end{itemize}
\end{frame}

\begin{frame}
  \frametitle{MPLS Fundamentals and Terminology}

  \begin{itemize}
  \item 
    %\pause
  \item 
  \end{itemize}
\end{frame}

\begin{frame}
  \frametitle{GMPLS}

  \begin{itemize}
  \item 
    %\pause
  \item 
  \end{itemize}
\end{frame}


\begin{frame}
  \frametitle{MPLS vs. GMPLS}

  \begin{itemize}
  \item 
    %\pause
  \item 
  \end{itemize}
\end{frame}


\begin{frame}
  \frametitle{GMPLS: Hierarchial LSP}

  \begin{itemize}
  \item 
    %\pause
  \item 
  \end{itemize}
\end{frame}




\begin{frame}
  \frametitle{Summary}
  \begin{itemize}
  \item MPLS
    %\pause
  \item GMPLS
  \end{itemize}
\end{frame}

\begin{frame}
  \frametitle{Resources}
  \begin{itemize}
  \item Generalized Multiprotocol Label Switching: An Overview of Signaling Enhancements and Recovery Techniques
IEEE Communication Magazine, July 2001.
A. Banerjee et. al. 
  \item Internet Traffic Engineering Using Multi-Protocol Label Switching (MPLS).
Computer Networks 40, Elsevier, 2002
D.O. Awduche and B. Jabbari. 
  \end{itemize}
\end{frame}

\end{document}
